\section*{Exercise 05: Drupal}
\subsection*{Background}
\subsubsection*{Which vulnerabilities do you think can be used? Pick two potential vulnerabilitiesand describe them in terms of why you picked them, i.e., date and exploit effect.}
\subsubsection*{For the rest of the tutorial, we will use the vulnerability \textit{dubbeddrupageddon.} What is the underlying vulnerability?}
\subsubsection*{What is so severe about the issue?}

\subsection*{Post-Exploitation}

\subsubsection*{What are possible activities/aims for the post-exploitation phase?}
\subsubsection*{Write out the list in the file that has the “User Accounts”?}
\subsubsection*{How does having a list of user names help?}
\subsubsection*{What do the excellent post exploitation scripts for linux offer?}

\subsection*{Reflection}

\subsubsection*{What is the main issue with the web server? How did it help selecting potential exploits?}
\subsubsection*{When opening the drupal web page, you are greeted by a warning. Do you think this is good practice? Why or why not?}
\subsubsection*{Given a more restrictive web server configuration, finding the relevant information wouldn't have been that easy. Please check dirbuster, to be found in the “Web Application Analysis” menu. How could this tool help you finding information? Try it outon the Ubuntu metasploitable VM. Use /usr/share/dirbuster/wordlists/directory-list-2.3-medium.txt as dictionary.}
\subsubsection*{How can effective spying with tools like dirbuster prevented?}
\subsubsection*{This attack didn't get us all the way to root. How would you continue the pentest? What would be your next actions?}
\subsubsection*{Do you have any specific things in mind you would try to get root access?}
\subsubsection*{What makes getting a remote shell so powerful?}