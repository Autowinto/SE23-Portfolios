\section{Exercise 02: Starting the Journey}
\subsection{Thinking About Threats}
Based on the three articles about the incident provided via the exercise, there are several things that can be said about the incidident.
\subsubsection{How did they separate access and infrastructure according to data relevance and impact?}
Prior to the Storm-0558 attack, Microsoft already had several security policies designed to limit access to data from unauthorized persons. Some of these were:

\begin{itemize}
    \item Employee background checks
    \item Employees had identifiable user accounts
    \item Strict access to workstations
    \item Multi factor authentication
    \item Requirements of regular password updates
\end{itemize}

In response to the Storm-0558 attack, Microsoft implemented several new security measures to avoid these types of attacks occurring in the future. These were:

\begin{itemize}
    \item Categorization of data and infrastructure elements according to severity and criticality
    \item Segregation of access and infrastructure based on aforementioned categorizations
\end{itemize}

These additional security implementations helped ensure, that an attack such as the Storm-0558 attack is much less likely to happen in the future.

\subsubsection{How do roles and personnel fit into this, and which role could policies and training play?}
Roles and personnel are integral to Microsoft's cybersecurity framework. Personnel are trained with regular refreshes to recognize security threats and respond to these accordingly. Microsoft has clear guidelines and protocols that employees must follow, which enhances security of the organization as a whole.

\subsection{Pentesting Intro}
\subsubsection{Which advantages for penetration testing would you see in the different approaches? What is the best option?}

\textbf{NAT Networks}\\
NAT networks allows multiple virtual machines to share a single network interface, effectively creating an isolated network sandbox, where the tester can perform their tests without impacting the external network, however, still allowing external communication if necessary.

\textbf{Bridged Networking}\\
Bridged networking is networking that connects a virtual machine to the actual network of the host machine, acting as a "bridge" allowing the Virtual Machine full access to the external network. The advantage of this, is as mentioned, full access to the external network, which is good when you're testing advanced scenarios that mimic real-life attacks.

\textbf{Host Only}\\
Host only is a network that completely isolates the virtual machine, disallowing networking with the external network. This is especially good when performing testing that requires isolation, such as testing malware or other things that we want to ensure doesn't go wrong.

\subsubsection{How does inspecting the ip configuration of a system help you with penetration testing? What is the security relevant aspect?}
It does so by giving you info about all internet adapters, their protocols, their addresses, metrics, etc. etc.

\subsubsection{How do you get the targeted user to execute our malicious payload?}
Social Engineering, Disguising the file, exploiting vulnerabilities that allow for automatic code execution.

\subsubsection{Is Metasploitable3 vulnerable to this exploit?}
Testing the vulnerability is simple as connecting to the metasploitable vm and accessing sysinfo, verifying if it's correct.
The vulnerability in this case, is an open nginx 8080 port, allowing us to connect.
Metasploitable3 is very vulnerable to this exploit as it's designed to be so.
It should close unused open ports, regularly update kernel and application versions, shut down unnecessary services and require validation before connection.
It's quite easy to trick someone to download malicious files through torrenting, limewire, linkin-park-in-the-end.exe etc.

\subsubsection{What is the practical use of this exercise? And why is the payload working in the way it is? How does this exercise relate to remote and reverse shells?}
The practical use of this exercise is to see how easy it is to gain access to a vulnerable systems shell. The payload works how it does because

\subsubsection{Which folder are you in when you get the meterpreter prompt? And whatis the system-information?}
I am in the folder that the payload.efi was run at

\subsubsection{As user and the owner of this system -- how would you mitigate this attack?}
By not chmodding and running payloads which I don't know what are smh.

\subsubsection{How does knowing usernames help an attacker/penentration tester?
}
It's a significant advantage as it allows you to brute-force passwords much faster and ensuring that you are actually on a user with specific permissions.

\subsubsection{Now that you have access to the Metasploitable machine what else can wedo? Get the list of users on this server, using a shell prompt by typing"shell" into the Meterpreter shell.}
TODO

\subsubsection{How does knowing usernames help an attacker/penentration tester?}
It's a significant advantage as it allows you to brute-force passwords much faster and ensuring that you are actually on a user with specific permissions.

\subsubsection{Using the meterpreter shell, check the output of the "arp" command. What do you find? Why could this information be relevant?
}
It displays internet-to-adapter address tables and when you're connected to a target machine, it shows the tables for that machine, which is very useful information when trying to penetrate.

\subsubsection{Now lets be on the other side of the fence and investigate suspicious connections to our metasploitable server. Which command can you use to see network status and connections? Is there an anomaly or suspicious connection to our server? What makes it suspicious?}
Unexpected source ip addresses, data transfers when you aren't expecting any, HTTP traffic on an unexpected port etc.