\section{Exercise 2: Starting the Journey}
\subsection{Thinking About Threats}
Based on the three articles about the incident provided via the exercise, there are several things that can be said about the incidident.
\subsubsection{How did they separate access and infrastructure according to data relevance and impact?}
Prior to the Storm-0558 attack, Microsoft already had several security policies designed to limit access to data from unauthorized people. Some of these were:

\begin{itemize}
    \item Employee background checks
    \item Employees had identifiable user accounts
    \item Strict access to workstations
    \item Multi factor authentication
    \item Requirements of regular password updates
\end{itemize}

In response to the Storm-0558 attack, Microsoft implemented several new security measures to avoid these types of attacks occurring in the future. These were:

\begin{itemize}
    \item Categorization of data and infrastructure elements according to severity and criticality
    \item Segregation of access and infrastructure based on aforementioned categorizations
\end{itemize}

These additional security implementations helped ensure, that an attack such as the Storm-0558 attack is much less likely to happen in the future.

\subsubsection{How do roles and personnel fit into this, and which role could policies and training play?}
Roles and personnel are integral to Microsoft's cybersecurity framework. Personnel are trained with regular refreshes to recognize security threats and respond to these accordingly. Microsoft has clear guidelines and protocols that employees must follow, which enhances security of the organization as a whole.

\subsection{Pentesting Intro}
\subsubsection{Which advantages for penetration testing would you see in the different approaches? What is the best option?}
It isn't possible to state a single best option for networking, as they all have their individual strengths and use cases.

\textbf{NAT Networks}\\
NAT networks allows multiple virtual machines to share a single network interface, effectively creating an isolated network sandbox, where the tester can perform their tests without impacting the external network, however, still allowing external communication if necessary.

\textbf{Bridged Networking}\\
Bridged networking is networking that connects a virtual machine to the actual network of the host machine, acting as a "bridge" allowing the Virtual Machine full access to the external network. The advantage of this, is as mentioned, full access to the external network, which is good when you are testing advanced scenarios that mimic real-life attacks.

\textbf{Host Only}\\
Host only is a network that completely isolates the virtual machine, disallowing networking with the external network. This is especially good when performing testing that requires isolation, such as testing malware or other potentially dangerous attacks.

\subsubsection{How does inspecting the ip configuration of a system help you with penetration testing? What is the security relevant aspect?}
Inspecting the ip configuration of a system helps us by giving you info about the configuration of the network, gateway and DNS information, whether the network uses IPv4 og IPv6, IP ranges and more. Generally, the more information you have about a network, the bigger the chance of there existing some sort of exploit that you can make use of.

\subsubsection{How do you get the targeted user to execute our malicious payload?}
You can attempt to have a user execute your malicious code with several different approaches. You can attempt social engineering to trick the user into believing that a file is completely harmless by exploiting their trust in you as a person or an organization you represent. You can attempt to disguise the file, making it look like a perfectly normal executable, a video file, a song or something entirely different, making the target lower their guard. Finally, you could potentially exploit existing automatic code execution exploits to run code without the user even knowing.

\subsubsection{What is the practical use of this exercise? And why is the payload working in the way it is? How does this exercise relate to remote and reverse shells?}
The practical use of this exercise is to see how easy it is to gain access to a vulnerable systems shell. The payload works the way it does, because in most realistic cases, there is not going to be an easily exploitable open connection that we can just connect to. We need to be let in by an incoming connection, in this case the payload, which opens up a connection for us that we can use. The exercise shows us how a remote shell works and how we can make use of it to control an external vulnerable machine.

\subsubsection{As user and the owner of this system -- how would you mitigate this attack?}
First of all, I would not use \texttt{chmod a+x} on random files that I was not sure were safe. chmod a+x gives permission for execution by all users, which is really not a very secure way of handling foreign files.

\subsubsection{How does knowing usernames help an attacker/penentration tester?}
It is a significant advantage as knowing a username allows you to begin brute-forcing the passwords of these users. In the case of a linux machine, the \texttt{/etc/passwd} file also contains information about which group a user is in, which if we have access to \texttt{/etc/groups} gives us the ability to figure out which users are super users, which in the case of penetration testing, are high-value targets.

\subsubsection{Using the meterpreter shell, check the output of the "arp" command. What do you find? Why could this information be relevant?}
Running the arp command on the metasploitable3 linux machine, gives us the following output.
\lstinputlisting{Outputs/E02/arp.txt}

It displays a table of ip addresses of the machine, the hardware addresses, interfaces and more, which is very useful information to have about a target machine that you are attempting to penetrate.

\subsubsection{Which command can you use to see network status and connections? Is there an anomaly or suspicious connection to our server? What makes it suspicious?}
We can use the \texttt{netstat} -a command to see all active connections and sockets. Something that makes a connection suspicious would be an unexpected source IP address, data transfers when you yourself aren't performing any and aren't expecting any and HTTP traffic on unexpected ports.