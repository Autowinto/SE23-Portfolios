\section{Concept Location}
Explain the methodology that you have used to locate each concept that was part of your change request.  Using Table 1, list all the files in the order that you have visited them (2nd column). Explain how you have found each file (3rd column). You can simply read the source code, but we encourage you to use the features provided by Netbeans and Featureous: “Quick Find”, “Find Symbol”, “Go To Definition”, “Call Browser”, “Find all References'', ”Class View”, “Insert Breakpoint”, or any other software tools that you want to use. 
Furthermore, Featureous Inspector is a good starting point for localizing domain classes based on their features.
In the fourth column, write what you have learned about the class. 

\begin{longtblr}[label = {tblr:domain}, caption = {The list of all the domain classes visited during concept location.}]{|l|l|l|l|}
    \hline
    \textbf{\#} & \textbf{Domain classes} & \textbf{Tool used} & \textbf{Comments} \\
    \hline
    \textbf{} & \textbf{} & \textbf{} & \textbf{} \\
    \hline
    \textbf{} & \textbf{} & \textbf{} & \textbf{} \\
    \hline
    \textbf{} & \textbf{} & \textbf{} & \textbf{} \\
    \hline
    \textbf{} & \textbf{} & \textbf{} & \textbf{} \\
    \hline
\end{longtblr}

Use Featureous Feature call-tree to provide a tree-based visualization of the runtime call graphs of methods implementing the features of your change request. This view provides an execution- based alternative to the hierarchical fashion of browsing features supported by the mentioned feature inspector view. 
