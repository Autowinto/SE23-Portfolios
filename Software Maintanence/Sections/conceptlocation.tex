\section{Concept Location}
\subsection{General Methodology}
In locating classed for the refactoring process of the report, there are several options available to us within Featureous and NetBeans, which we of course will make use of in the future impact analysis section of the report. For actually finding the location of classes relevant to the feature, I mainly made use of the powerful tools provided by the IntelliJ IDE, such as the following:

\begin{itemize}
    \item Find usages - Allows you to find usages of any given method or variable
    \item Go to implementation - Allows you to go directly to the implementation of a method
    \item Search within file - Allows you to search for any symbol or token within a single file.
    \item Inspect code - Allows you to see a bunch of metrics about code such as maturity, implementation issues and more, helping to pinpoint problematic parts of the code.
\end{itemize}

In addition to tooling provided by NetBeans and IntelliJ, there are also methods that are much less technical, such as simply removing parts of the code and seeing what happens and if said code is relevant to the feature that is being refactored.

\subsection{Classes Overview}

\begin{center}
    \begin{tblr}{hlines, vlines}
        \bf{\#} & \bf{Domain classes}       & \bf{Tool used} & \bf{Comments} \\
        1       & LoadFileAction            &                &               \\
        2       & OpenFileAction            &                &               \\
        3       & AbstractApplicationAction &                &               \\
        4       & AbstractAction            &                &               \\
    \end{tblr}
\end{center}

Use Featureous Feature call-tree to provide a tree-based visualization of the runtime call graphs of methods implementing the features of your change request. This view provides an execution- based alternative to the hierarchical fashion of browsing features supported by the mentioned feature inspector view.
