\documentclass{article}
    % General document formatting
    \usepackage[margin=0.7in]{geometry}
    \usepackage[parfill]{parskip}
    \usepackage[utf8]{inputenc}
    \usepackage{tabularray}

\title{Opening various image formats}



\begin{document}
    \textbf{Name: Magnus Christian Larsen\\
        GitHub Username: Autowinto\\
        Student Mail: magla21@student.sdu.dk\\
        Project: https://github.com/Autowinto/JHotDraw\\
        Date: mm/dd/yyyy
    }
\newpage
\tableofcontents
\newpage

\begin{center}
    \huge Opening various image formats 
\end{center}

\section{Change Request}
In this section, describe the change that you were requested to do.  Give any relevant background information or any essential details.
Remember your change request has to represent a functional requirement (feature) and has to be described as a User Story.

\section{Concept Location}
Explain the methodology that you have used to locate each concept that was part of your change request.  Using Table 1, list all the files in the order that you have visited them (2nd column). Explain how you have found each file (3rd column). You can simply read the source code, but we encourage you to use the features provided by Netbeans and Featureous: “Quick Find”, “Find Symbol”, “Go To Definition”, “Call Browser”, “Find all References'', ”Class View”, “Insert Breakpoint”, or any other software tools that you want to use. 
Furthermore, Featureous Inspector is a good starting point for localizing domain classes based on their features.
In the fourth column, write what you have learned about the class. 

\begin{longtblr}[label = {tblr:domain}, caption = {The list of all the domain classes visited during concept location.}]{|l|l|l|l|}
    \hline
    \textbf{\#} & \textbf{Domain classes} & \textbf{Tool used} & \textbf{Comments} \\
    \hline
    \textbf{} & \textbf{} & \textbf{} & \textbf{} \\
    \hline
    \textbf{} & \textbf{} & \textbf{} & \textbf{} \\
    \hline
    \textbf{} & \textbf{} & \textbf{} & \textbf{} \\
    \hline
    \textbf{} & \textbf{} & \textbf{} & \textbf{} \\
    \hline
\end{longtblr}

Use Featureous Feature call-tree to provide a tree-based visualization of the runtime call graphs of methods implementing the features of your change request. This view provides an execution- based alternative to the hierarchical fashion of browsing features supported by the mentioned feature inspector view. 

    
\section{Impact Analysis}

Use Featureous Feature-Code Characterization View to illustrate Scattering and Tangling of each feature that is involved in your Change Request. 
Write your reflections about how these measurements relate to your feature and future refactorings. For example, text about the purpose of Impact Analysis and ripple effects.

Use Featureous Feature Relations Characterization view to relate your change request features to each other with respect to how they depend on the objects created by other features and how they exchange data by sharing these objects with one another.

Use Feature-code correlation graph and feature-code correlation grid to illustrate detailed investigations of the correspondences between source code units and related features to your change request.

Using table 2 list the packages and their number of classes that you visited after you located the concept. Write short comments explaining what you have learned about each package and how they contribute to your feature?

In the fourth column, mention if the class is related to the concept. Use one of the following terms:
\begin{itemize}
    \item Use “Unchanged” if the class has no relation to the concept but you have visited it. 
    \item Use “Propagating” if you read the source code of the class and it guides you to the location of the concept, but you will not change it.
    \item Use “Changed” if the class will be changed.
\end{itemize}


\begin{longtblr}[caption = {The list of all the packages visited during impact analysis.}]{|l|l|l|l|}
    \hline
    \textbf{Package name} & \textbf{\# of classes} & \textbf{Tool used} & \textbf{Comments} \\
    \hline
    \textbf{} & \textbf{} & \textbf{} & \textbf{} \\
    \hline
    \textbf{} & \textbf{} & \textbf{} & \textbf{} \\
    \hline
    \textbf{} & \textbf{} & \textbf{} & \textbf{} \\
    \hline
    \textbf{} & \textbf{} & \textbf{} & \textbf{} \\
    \hline
\end{longtblr}

\section{Refactoring Patterns and Code smells}
Describe the code smell that triggered your refactoring, see Chapter 4 in [Ker05]. 
Describe what you plan to change by refactoring. 
Why do you think it's good to do the refactoring?
Describe the strategy of the refactorings. Remember there is not one way to implement a pattern.
Which of the refactorings from [Ker05] did you apply and what was the reasoning behind it?

\section{Refactoring Implementation}
Explain where and why you made changes in the source code.  That is, explain the difference between the actual change set compared to the estimated impacted set from the impact analysis. Which classes or methods did you create or change? Furthermore, describe used design patterns and reflect about improved design to improve future software maintenance.

\section{Verification}
At class level document unit tests of important business functionality.
Document how you have verified your implemented change. 
Document the results of your acceptance test that test your feature from your change request.

\section{Continuous Integration}
Reflect on the following questions:
\begin{itemize}
    \item What is Continuous Integration and how could you apply it?
    \item How to use version control systems (git) based on your portfolio?
\end{itemize}

\section{Conclusion}
Explain your experience with creating the final baseline of JHotDraw.
How did you manage to merge in your changes to the baseline ?
How was the system tested?
Describe your reflection on what went well and what went not so well. 
Did you have to cut the scope of your change request and did you have to put some issues back into the backlog. 
What can be done to avoid future problems, i.e. what have you learned from the iteration.

\section{Source Code}
Submit all your code changes to your local Git repository and finally the remote GitHub repository. 
Please provide links to your source code / Feature Branch. 

\bibliographystyle{plain}
\bibliography{References}

\end{document}