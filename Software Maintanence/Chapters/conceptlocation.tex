\section{Concept Location}
\subsection{General Methodology}
In locating classed for the refactoring process of the report, there are several options available to us within Featureous and NetBeans, which we of course will make use of in the future impact analysis section of the report. For actually finding the location of classes relevant to the feature, I mainly made use of the powerful tools provided by the IntelliJ IDE, such as the following:

\begin{itemize}
    \item Quick find - Looking through the code manually, searching for relevant keywords and such.
    \item Find usages - Allows you to find usages of any given method or variable
    \item Go to implementation - Allows you to go directly to the implementation of a method
    \item Search within file - Allows you to search for any symbol or token within a single file.
    \item Inspect code - Allows you to see a bunch of metrics about code such as maturity, implementation issues and more, helping to pinpoint problematic parts of the code.
\end{itemize}

In addition to tooling provided by NetBeans and IntelliJ, there are also methods that are much less technical, such as simply removing parts of the code and seeing what happens and if said code is relevant to the feature that is being refactored.

\subsection{Classes Overview}
\begin{longtblr}{hlines, vlines, colspec={l l l X[j]}}
    \bf{\#} & \bf{Domain classes}              & \bf{Tool used}       & \bf{Comments}                                                                                                                                                                                                                           \\
    1       & LoadFileAction                   & Quick find           & I took a look at the codebase as a whole, scanning through the different files and classes and quickly idscovered LoadFileAction, which has a name easily identifiable to be relevant to the feature that is to be refactored.          \\
    2       & OpenFileAction                   & Quick find           & Again, while taking a look through the codebase I found the OpenFileAction class which made sense to include as a relevant class.                                                                                                       \\
    3       & AbstractSaveUnsavedChangesAction & Go to implementation & The AbstractSaveUnsavedChangesAction class was found by finding the implementation of the class that LoadFileAction extends.                                                                                                            \\
    3       & AbstractApplicationAction        & Go to implementation & The AbstractApplicationAction is extended by the OpenFileAction                                                                                                                                                                         \\
    3       & AbstractViewAction               & Go to implementation & The AbstractViewAction is extended by the AbstractSaveUnsavedChangesAction class.                                                                                                                                                       \\
    3       & AbstractAction                   & Go to implementation & Both the AbstractApplicationAction and the AbstractViewAction extends the AbstractAction class, which is worth looking into, as it doesn't make a lot of sense why there are so many different abstract classes that are extended from. \\
    4       & URIChooser                       & Go to implementation & The URIChooser is a class used by both the LoadFileAction and the OpenFileAction and has their own implementation in each class. It doesn't necessarily make much sense for two almost identical implementations to exist.
\end{longtblr}
