\section*{Contributions}
My contributions to the project lie mainly in the implementation of the application, as well as creation of the StyleSheets used to ensure that the design and the actual application were design-wise aligned. The project as a whole was setup and managed by me in GitHub repository on my account. Having a look at the insights tab in the \href{https://github.com/Autowinto/MobileSoftApp}{git repository}, will also reveal that a majority of the commits in the repository was performed by me.

Additionally, a lot of the research of how to actually make use of React Native and Expo Go was done by me, which I spent a lot of work communicating said research to other group members.


\subsection*{Views \& Components}
I did a great deal of implementation work in the views, components and types of the application.
I created the base views for every page and was one of the main developers of the Search.tsx, SearchResults.tsx, UserProfile.tsx, as well as coming up with the idea for having a separate NavBar.tsx file that could be reused across pages.

A lot of work that I did went into refactoring components that were very messy and reused logic into reusable components and with much neater code. The CarItem.tsx and CarListItem.tsx are direct results of my contributions in that regard.

I also revised a lot of the application based on feedback received, namely implementing more user feedback when cars were rented, adding a popup letting the user know that the action was succesful and navigating back to the CarList page, now showing updated data.


\subsection*{Utilities}
The source code has a utils folder, holding the data.ts file. This utility file was created by me and holds logic for fetching and managing api data. In our application, we fetch api data once, store it in the AsyncStore and update this local data, allowing us to have persistent data in our demo application. The data that is fetched is courtesy of another student who created a simple car api.

\subsection*{Styling}
In terms of stylesheets, another group member created many of the base styles and concepts that we used to translate our figma model into an actual application, I did however to a lot of custom styling on the Car List screen as well as some on the Login page to ensure consistency in the design.

While at first the StyleSheets were well-organized, as the application grew, the StyleSheets became much more disorganized and increasingly strayed from the actual figma design, which, in a future development process, would be something to look out for.